\subsection{Description/additional circuitry}
Car is using two pressure sensors as encoders for brake pedal. Each one is for separate brake system. Sensors are powered and measured by \gls{ecup}, same unit as \ref{sec:TorqueEncoder}. Pressure sensors output analog voltage signal equal to brake system pressure. 

%\def\tabularxcolumn#1{m{#1}}

\begin{table}[H]
	\centering
	\caption{Brake encoder data}
	\begin{tabularx}{\textwidth}{|X|X|}
		\hline
		Torque encoder manufacturer: &  Keller \\[\TableSize]\hline
		Torque encoder type: & PA-21 Y \\[\TableSize]\hline
		Torque encoder principle: & pressure sensor \\[\TableSize]\hline
		Total number of Torque Encoder Sensors: & 2 \\[\TableSize]\hline
		Supply voltage: & +24 V \\[\TableSize]\hline
		Maximum supply current: &  4 mA  \\[\TableSize]\hline
		Operating temperature: & -40 $^\circ$ to 100 $^\circ$C \\[\TableSize]\hline
		Used output: & DC voltage 0.5 V to 4.5 V \\[\TableSize]\hline
	\end{tabularx}%
	\label{tab:brake-general}%
\end{table}%

\begin{figure}[H]
	\begin{center}
		\includegraphics[width=\textwidth]{./img/BRK-pos.jpg}
		\caption{Pressure sensor positon.}
		\label{fig:brake_pressure_position}
	\end{center}
\end{figure}

\subsection{Brake Encoder Plausibility Check}
Implausibility is not checked, because eForce does not use brake pedal information to control \glspl{mc}.

%
%\subsection{Wiring}
%Describe the wiring, show schematics, show data regarding the cables and connectors used.
%
%\subsubsection{Position in car/mechanical fastening/mechanical connection}
%Provide CAD-renderings showing all relevant parts and discuss the mechanical connection of the sensors to the pedal assembly. Mark the parts in the rendering, if necessary.




