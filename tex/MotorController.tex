\subsection{Motor Controller 1}
% Kabeláž
% add schematic and datasheet

\subsubsection{Description, type, operation parameters}
%Describe important functions; provide table with main parameters like resulting voltages->minimum, maximum, nominal, currents etc.
Motor controller is a prototype of MiRy X-Boss Motor Controller. It is fully self-designed for driving 2 \gls{pmsm} motors simultaneously with Resolver sensors as a position feedback. Field Oriented Control is implemented with galvanic isolated current sensors LEM HTFS 200-P.

Galvanic isolation on PCB is shown in \ref{app:mc-top} and \ref{app:mc-bot}. The only place where isolation is smaller then required 4mm is between pins of DC/DC NME1215SC. Real space gap is 1.54 mm. The NME1215SC has rated voltage of 1kV. See \ref{app:NME1215SC}.

Motor Controller communicate with traction control unit by private CAN bus. If any error occurs in their communication – Motor Controller stops driving both motors (error mode). It also implements discharge circuit which is activated by CAN or in case of Auxiliary supply disconnection (resistors are driven by normally-closed relay).

A current limit is set to not overload used motors. For each motor controller different. Rear motor controller has set peak current limit to 202 A and temperature limit to 120 \degC. Front Motor Controller’s current limit is 70 A and temperature limit also 120 \degC.

\begin{table}[H]
	\centering
	\caption{General motor controller data}
	\begin{tabu}{|X|X|}\hline
		Motor controller type: & MiRy X-Boss \\\hline
		Maximum continuous power: & 2 x 90 kW (in:400 V, out:200 A) \\\hline
		Maximum peak power: & 2 x 146 kW for 5s (in:400 V, out:300 A) \\\hline
		Maximum Input voltage: & 410 \vdc \\\hline
		Output voltage: & 282 \vac \\\hline
		Maximum continuous output current: & 2 x 200 A \\\hline
		Maximum peak current: & 2 x 300 A for 5s \\\hline
		Control method: & CAN \\\hline
		Cooling method: & Water \\\hline
		Auxiliary supply voltage: & 24 \vdc \\\hline
	\end{tabu}%
	\label{tab:MC:general}%
\end{table}%

\subsubsection{Wiring, cables, current calculations, connectors}
%Describe the wiring, show schematics, provide calculations for currents and voltages and show data regarding the cables and connectors used.

Motor controllers are connected with \gls{hvd} box by high voltage cable. High current connector “ASHD 0 22-24320 P N” by Deutsch is used. 

\begin{table}[H]
	\centering
	\caption{Wire data of OLFLEX HEAT 180}
	\begin{tabu}{|X|X|}\hline
		Wire type: & OLFLEX HEAT 180 SiF  \\\hline
		Current rating: & 165 A \\\hline
		Fuse current rating: & 160 A \\\hline
		Maximum operating voltage: & 500 V \\\hline
		Temperature rating: & 180 \degC \\\hline
	\end{tabu}%
	\label{tab:MC:wire}%
\end{table}%

%table
\subsubsection{Position in car}
%Provide CAD-renderings showing the relevant parts. Mark the parts in the rendering, if necessary.

The position of both motor controllers is shown in \ref{fig:MC:position}.

\begin{figure}[H]
	\centering
	\includegraphics[width=\textwidth]{./img/MC-position.jpg}
	\caption{MC position.}
	\label{fig:MC:position}
\end{figure}
\subsection{Motor Controller 2}
%If identical parts are used, just refer to the corresponding sections, don’t copy and paste.
Motor controller 2 is identical to Motor controller 1.




