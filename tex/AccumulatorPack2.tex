\subsubsection{Description}

The GLVS Accumulator is built out of 6 rechargeable Lithium-Ion cells connected in series. It shares a Kevlar enclosure with ECU-B, which provides the required fire resitance. The battery is secured to the enclosure using 3M Dual Lock fasteners.

The accumulator provides GLVS power at startup until the HV-to-LV DCDC Converter takes over. Temperature and voltages are monitored using an XXXXX IC. Voltages are measured by the IC directly and temperatures are measured on N NTC thermistors.

\subsubsection{Wiring, cables, current calculations, connectors}
Describe wiring, show schematics, describe connectors and cables and show useful data regarding the wiring.  Include information on the working voltage and current rating of the accumulator.
TODO: connector XT-60, fuse atd.

\begin{table}[H]
	\centering
	\caption{GLVS accumualtor general parameters.}
	\begin{tabularx}{\textwidth}{|X|X|}\hline
		Cell/Accumulator: & SONY US18650VTC5A (Lithium-Ion)\\[\TableSize]\hline
		Accumulator configuration – parallel: & 1 \\[\TableSize]\hline
		Accumulator configuration – series: & 6 \\[\TableSize]\hline
		Maximum Voltage: & 25.5 V \\[\TableSize]\hline
		Nominal Voltage: & 21.6V \\[\TableSize]\hline
		Minimum Voltage: & 12V \\[\TableSize]\hline
		Max. Continuous Discharge Current: & 35 A \\[\TableSize]\hline
		Peak Discharge Current: & 40 A \\[\TableSize]\hline
		Peak Discharge Current Time: & 78 seconds \\[\TableSize]\hline
		Max. Continuous Charge Current & 6 A \\[\TableSize]\hline
		Total capacity[MJ]: & 0.2387 \\[\TableSize]\hline
	\end{tabularx}%
	\label{tab:LVbatt-general}%
\end{table}%

\subsubsection{Position in car}
Provide CAD-renderings showing all relevant parts. Mark the parts in the rendering, if necessary.
