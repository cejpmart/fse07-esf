\subsection{Description/additional circuitry}
Car is using two torque encoders for accelerator pedal. Encoders are linear potentiometers. Sensors are powered and measured by \gls{ecup}. They output analog voltage signal equal to accelerator pedal position. Both output signals goes through \ref{fig:ecup_analog_input} and low pass filters. Then they are fed to \gls{adc} and analyzed by \gls{mcu} for validity and plausibility. Information about position and errors is send though \gls{can}.

\begin{table}[H]
	\centering
	\caption{Torque encoder data}
	\begin{tabu}{|X|X|}
		\hline
		Torque encoder manufacturer: & TE connectivity  \\\hline
		Torque encoder type: & MLP-50  \\\hline
		Torque encoder principle: & linear potentiometer  \\\hline
		Total number of Torque Encoder Sensors: & 2  \\\hline
		Supply voltage: & +5 V  \\\hline
		Maximum supply current: & 1 mA  \\\hline
		Operating temperature: & -30 \degC to -150 \degC  \\\hline
		Used output: & DC voltage 0 V to 5 V\\\hline
	\end{tabu}%
	\label{tab:encoder-general}%
\end{table}%

Full datasheet: \ref{app:torque_encoder_datasheet}

\subsection{Torque Encoder Plausibility Check}
Torque encoder implausibility, short circuit and open circuit checks are done by \gls{mcu}. Sensors are powered from +5V. Used travel of sensors does not contain end positions. Converted analog signal ranges are normalized according to calibration.
If any following sensor error is detected, motor controllers are shut down and message is send through \gls{can} to other units.

\paragraph{Electrical check} 
Using only part of whole range of sensor excluding end positions allows detection of signal short to Ground or voltages higher or equal to sensor's power voltage.

Analog input circuitry of \gls{ecup} is shown in \ref{fig:ecup_analog_input}. In normal conditions, signal from sensor (RAW) goes through $R_2$, then is clamped by $D_1$ and continues through $R_1$ and $C_1$ to filters and other circuits of \gls{ecup} (AIN). TEST signal is held \textit{low} but due to high resistance of $R_3$ has minimal impact on input signal.

When \gls{adc} measures 0V or value close to 0V, TEST signal logic level is switched \textit{high} to charge $C_1$ through $R_3$. After short while, logic level of TEST signal is read back. If \textit{high} logic level is read, sensor input is evaluated as \textit{open circuit}, otherwise it is \textit{short circuit} to GND. If measured value is close to power voltage of sensor or above measurable range, it is evaluated as \textit{short circuit} to power voltage or higher.

To recognize short between signals of both sensors (rule \textit{T 10.3.6}), one potentiometer has resistor in series to achieve different transfer function (gradient). 

\paragraph{Implausibility}
Position values difference is calculated and compared with maximal allowed error threshold (10\%).

\begin{figure}[H]
\begin{center}
	\includegraphics[width=0.5\textwidth]{./img/ECUP_AIN.pdf}
	\caption{\gls{ecup} analog input circuit}
	\label{fig:ecup_analog_input}
\end{center}
\end{figure}



\subsection{Wiring}
\ref{fig:ecup_wiring} is diagram of wiring between accelerator pedal sensors and \gls{ecup}.

\begin{figure}[H]
\begin{center}
	\includegraphics[width=0.5\textwidth]{./img/ECUP_wiring.pdf}
	\caption{ECU-P sensor wiring}
	\label{fig:ecup_wiring}
\end{center}
\end{figure}

\subsection{Position in car/mechanical fastening/mechanical connection}
In \ref{fig:torque_encoder_position} is shown position of two torque encoder sensors (piston shape objects).

\begin{figure}[H]
\begin{center}
	\includegraphics[width=\textwidth]{./img/ACC-pedal-pos.jpg}
	\caption{Torque encoder position}
	\label{fig:torque_encoder_position}
\end{center}
\end{figure}



%\begin{figure}[H]
%\begin{center}
%	\includegraphics[width=0.8\textwidth]{./img/ACC-pedal-pos2.jpg}
%	\caption{Torque encoder position}
%	\label{fig:torque_encoder_position_alt}
%\end{center}
%\end{figure}


